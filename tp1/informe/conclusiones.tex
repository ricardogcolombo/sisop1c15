Como concluciones para este trabajo practico podemos decir que el scheduler es algo muy dependiente de que estemos intentando lograr con el. Vimos que para siertos casos un round rovin puede darnos un buen waiting time y un buen turaround, siempre y cuando el quantum sea adecuado y la migracion de procesos entre CPUs sea despreciable. En caso de no serlo, el Round Rovin puede obtener mejores reslutados siempre y cuando no se presenten tareas patologicas que produzcan que un procesador este sobrecargado y otro no tenga tareas para ejecutar.
\\
Por otra parte si lo que se busca es un scheduler de tiempo real que cumpla con deadlines estrictas, puede optarse por alguno de los dos algoritmos descriptos en la segunda parte del trabajo practico ya que nos asegurarán que estas dedlines se cumplan.
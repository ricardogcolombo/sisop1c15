\documentclass[a4,11pt]{article}

\parindent=10pt
\parskip=6pt
%\usepackage[width=15.5cm, left=2.5cm, top=2cm, height= 24.5cm]{geometry}
\usepackage[paper=a4paper, left=2cm, right=2cm, bottom=2.5cm,top=2.5cm]{geometry}

% Paquetes de nacionalización. No olvidar para poder poner tildes!
\usepackage[spanish]{babel}
\usepackage[utf8]{inputenc}

% Paquetes para graficos
\usepackage{subfig}
% \usepackage{graphicx} %% La caratula lo incluye

% Paquetes para matematica
\usepackage{amsmath}
\usepackage{amsfonts}
\usepackage{amssymb}

% Paquetes para pseudo
\usepackage{algorithm}
\usepackage{algorithmic}

% Caratula (Recordar logo_uba.jpg y logo_dc.jpg)
\usepackage{caratula}

% Paquetes para tablas
\usepackage[table]{xcolor}

% Se pueden sacar?
\usepackage{url}
\usepackage{float}
\usepackage{afterpage}
\usepackage{tabularx}

% Color de links
\usepackage{hyperref}
\hypersetup{
    colorlinks,
    citecolor=black,
    filecolor=black,
    linkcolor=black,
    urlcolor=black
}

\begin{document}


\materia{Metodos numericos}
\submateria{Primer Cuatrimestre de 2015}
\titulo{Trabajo Pr\'actico 2}
\subtitulo{“pthreads”}
\integrante{Ricardo Colombo}{156/08}{ricardogcolombo@gmail.com}
\integrante{Franco Negri}{893/13}{franconegri2004@hotmail.com}
\integrante{Federico Suarez}{610/11}{elgeniofederico@gmail.com}

\maketitle
\pagebreak
  
\tableofcontents

\pagebreak

\section{Algoritmos Servidor}

\subsection{Envío una letra}
Una vez que se identifica que el mensaje recibido es una letra, se chequea que ese mensaje esté bien formado, es decir, que seguido de la palabra LETRA se reciban también 3 valores adicionales correspondientes a la fila, columna y letra a colocar.
En este punto se pide un WriteLock sobre el tablero de letras, previo a validar si puedo escribir la letra en el tablero.
Luego se hacen las validaciones que respectan al juego: que no esté ocupado el lugar a colocar, que no haya espacios en el medio de la palabra que se está formando, y que no se cambie la orientación de la palabra de vertical a horizontal o viceversa.
Si se pasan estos controles, la letra efectivamente se escribe en el tablero de letras, se agrega a mi lista de letras que representa la palabra actual que se está escribiendo y se envía un mensaje de OK.
En caso de no cumplirse alguna de las validaciones, se quitan todas las letras que conforman la palabra actual del tablero, se limpia la lista de letras de la palabra actual y se envía un mensaje de ERROR.
Finalmente se libera el lock.

Si ocurre algún error de comunicación, se libera el lock y termina el cliente.

\subsection{Confirmación de una palabra}

Si el mensaje recibido es la confirmación de una palabra entonces todas las letras pertenecientes a la palabra actual que están escritas en el tablero de letras se pasan al tablero de palabras y luego se limpia la lista de letras que representa la palabra confirmada y se envía un mensaje de OK.
Previo a realizar esta operación, se pide un WriteLock sobre el tablero de palabras que una vez finalizada la escritura se libera.
Si ocurre algún error de comunicación, se libera el lock y termina el cliente.


\subsection{Petición de Actualización}

Si se recibe un mensaje de UPDATE, primero se realiza un ReadLock previo a la lectura del tablero de palabras. Luego efectivamente se lee el tablero de palabras ya confirmadas y se lo envía con el formato establecido.
Finalmente se libera el lock.
Si ocurre algún error de comunicación, se libera el lock y termina el cliente.

\end{document}
